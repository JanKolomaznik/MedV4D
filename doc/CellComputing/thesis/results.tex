\chapter{Results}

\section{Speed measurements}
//TODO popsat ze sem musel zmensit dataset

\begin{table}
\centering
\begin{tabular}{|c|c|c|c|}
\hline
\multicolumn{4}{|c|}{Measurement results}\\
\hline
Data set&Architecture&time spend&percent\\&&in calculations&\%\\&&(in seconds)&\\
\hline
\hline
ApplyUpdate()	&				&	20.15&	75.21\\
\hline
		&PropagateAllLayerValues()	&	16.64&	62.11\\
\hline
		&UpdateActiveLayerValues()	&	2.27&	8.47\\
\hline
CalculateChange()&				&	6.11&	22.8\\
\hline
		&ComputeUpdate()		&	3.97&	14.82\\
\hline
TOTAL		&				&	26.79&	100\\
\hline
\end{tabular}
\par
\caption[Measurement results]
{
  Results of speed measurement //TODO
}
\label{tab:runresults}
\end{table}

\section{Pictures}

\section{Reasons of slowdown and possible improvements}
//popsat rezii prenaseni okoli, na todle ze to je naprd
// popsat jake jsou moznosti na zrychleni a co proto je treba udelat.

\section{Code and design complexity}
// porovnat slozitost kodu a navrhu

\section{Algorithm complexity}
// popsat proc todle neni algoritmus, ktery je krasne zrychlitelny a uvest protipriklad. Treba tresholding?
\par
The implemented algorithm is complex enough.
So its worth to let it to compute on a remote machine with Cell B.E. (or any other processor).
It is tradeoff between time spent in transfer of actual dataset and the time we spare with computation on better hardware.
I.e. simple thresholding would be worthless to compute remotely because

But the nature of the thresholding is exactly targeted for streaming architecture such as Cell B.E..
So when the Cell B.E. processor is in common machines such as notebooks, desktops, even class of such simple image processing algorithms (such as the thresholding or variety of masking, edge detections) that is implemented in everyday used software could take advantage of the processor.
But nowadays is available only in Blade servers and PS3.
So only complex algorithm is worth to port.
We think the pallet of tools and features of the Cell B.E. can make it well suited for every algorithm.
Some of the algorithms suits more and the performance gain is huge.
But some of them could not suit as well and performance gain would be less but we don't think there is an algorithm that can perform much worse (i.e. have big negative performance gain).

\chapter{Conclusion}

At the beginning we studied available literature to find out what is actually Cell B.E., what benefits it brings for what price.
What special features it have and what are they good for.
Then we have been trying to install SDK to start actual development process.
During this phase we faced some obstacles like bugs, incompatibilities among tools, libraries that the tools use and even operating system vs. SDK incompatibilities.
So we had to go through variety of forums and other sources to find the solution.
As a side effect we improved our Linux knowledge.
Eventually we managed to install SDK and was able to start developing.
Then we tested variety of libraries, tools and other feature that the SDK brings.
We have chosen level set based segmentation algorithm implementing sparse field accelerating method to port to Cell B.E. platform.
This is quite complex algorithm to test the platform's potential.
We adopted ITK implementation of that algorithm.
So we had to study ITK toolkit and its internals.
We have also incorporated the whole program into MedV4D framework.
That means we have implemented some new modules that allows using ITK and can offload some part of processing to another machine (that run on Cell B.E.).
Actual porting process started with profiling of existing application.
This step have found out hot spots of the code which can be in turn offloaded to SPE to take advantage of Cell B.E. potential.
But results was qiote unexpected so another redesign of application followed.
In this new design almost whole original ITK pipeline was offloaded into SPE.
Big code restructuralisation was necessary to allow to perform actual computations on SPE.
Finnaly we have able to run the whole algorithm on SPE and thus to measure time need for computations.
The result of measurement showed that simple move the computation to SPE slows down the computations so when one want to take advantages of Cell B.E. potential another code optimalisations should be performed.
We have proposed some optimalisations that could increase the performance of our application.

\par
The Cell B.E. platform is very interresting for its variations of use scenarios and ability of program tuning and customization.
We think its great potential has already been proven.
But it is still waiting for wider spectrum of programmers.

\par
If the process of starting developing on Cell B.E. would become simplier we belive much more new programmers would be start.
Nowadays there are plenty of information about Cell B.E. but somehow unsorted or out of date.
The best information source are documents shipped along the SDK.
But they are targetted to contain all the information regardless the level of experience of the reader.
So when a programmer wants to start developing applications on Cell B.E. he would go trough plenty of that information before he can start actual work.
It's a pity there are total lack of information for PS3 users within SDK docs.
This is quite problem when big part of begginers has PS3 available.
There is simply lack of some "cookbook for beginners" with practical information and some howtos.
We believe is such cookbook with some of practical information that potentially may help to some other programmers who would like to start developing for Cell B.E.