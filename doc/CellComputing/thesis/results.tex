\chapter{Results}
Tabulka:::
\begin{center}
\begin{tabular}{|l|c|p{3.5in}|}
\hline
\multicolumn{3}{|c|}{Nazev tabulky}\\ 
\hline cell 1&cell 2&cell 3\\&todle bude pod cell 2, protoze to je mezi dvema'\&' &\\ 
\hline Big Basin&1.5&Very nice overnight to Berry Creek Falls from
either Headquarters or ocean side.\\ 
\hline Sunol&1&Technicolor green in the spring. Watch out for the cows.\\ 
\hline Henry Coe&1.5&Large wilderness nearby suitable for multi-day treks.\\ 
\hline
\end{tabular}
\end{center}

\chapter{Conclusion}

At the beginig I studied avalable literature to find out what is actualy
CellBE, what benefits it brings for what price. What special features it have
and what are thay good for. Then I have been trying to install SDK to start
actual development process. During this phase I faced some obstacles like bugs,
incompatibilities among tools, lilbraries that the tools use and even
inoperating system vs. SDK incompatibilities. So I had to go through
variety of forums and toher sources to find the solution. As a side
effect I improved my linux knowledge. Eventually
I managed to install SDK and was able to start developing. Then I tested
variety of libraries, tools and other feature that the SDK brings.
I have chosen level set based segmentation algorithm implementing sparse field
accelerating method to
port to CellBE platform. This is quite complex algorithm to test the platform
potential for image processing as a whole. I adopted ITK implementation of that
algorithm. So I had to study ITK toolkit and its internals. I have also
incorporated the whole program into MedV4D framework that means I have
imiplmented some new modules that allows using ITK and can offload some part of
processing to another machine (that run on CellBE).
Actual porting process started with profiling of existing application which had
find out hot spots of the code which can be in turn offloaded to SPE to //TODO
vyuzit CellBE potential.
Profiling step results are unexpected so another redesign of application
followed. In this new design almost whole original ITK pipeline was offloaded
into SPE. Big code restructuralisation was necessary to allow to perform actual
computations on SPE.



The implemented algorithm is complex enough. So its worth to let it to compute
on a remote mashine with CellBE (or any other processor). I.e.
simple thresholding would be worthless to compute remotely. It is tradeoff
between time spent in
transfer of actual dataset and the time we spare with computation on
better hardware. But the nature of the thresholding is exactly targeted for
streaming arghitecture such as CellBE. So when the CellBE processor is in common
machines such as notebooks, desktops, even class of such simple image processing
algorithms (such as the thresholding or variety of maskings, edge detections)
that is implemented in everyday used software could take advantege of the
processor. But nowadays is available only in Blade servers and PS3. So only
complex algorithm is worth to port. I think the pallete of tools and features
of the CellBE can make it well suited for every algorithm. Some of the
algorithms siuts more and the performance gain is huge. But some of them could
not suit as well and performance gain wouldbe less but I dont think there is an
algorithm that can perform much worse (i.e. have big negative performance gain).

Personaly I like the CellBE platform for its varations of use scenarios
and ability of program tunnig and customization. I think its great potential has
already been proved. But it is still waiting for wider spectrum of programmers.

// popsat, ze by bylo lepsi, kdyby to bylo jednodussi