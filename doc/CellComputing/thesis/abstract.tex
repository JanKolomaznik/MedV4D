\raggedbottom


%%%%%%%%%%%%%%%%%%%%%%%%%%%%%%%%%%%%%%%%%%%%%%%%%%%%%%%%%%%%%%%%%%%%%%%%%%%%%%%%
\noindent
\textbf{Název práce}: \textit{Implementace algoritmů pro zpracování obrazu na IBM Cell} \\
\textbf{Autor:} \textit{Bc. Václav Klecanda} \\
\textbf{Katedra (ústav):} \textit{Kabinet software a výuky informatiky} \\
\textbf{Vedoucí diplomové práce:} \textit{Mgr. Václav Krajíček} \\
\textbf{e-mail vedoucího:} \textit{Vaclav.Krajicek@mff.cuni.cz} \\
\textbf{Abstrakt:} \\

\par
\noindent
\textit{
Práce shrnuje dostupné informace o architektuře IBM \mbox{Cell/B.E.} tak, aby čtenář rychle získal potřebný náhled na problematiku programování pro tuto architekturu.
Praktické informace jsou čerpány z vývoje aplikace která implementuje netrivialní algoritmus z oblasti zpracování obrazu, sparse field level set segmentation.
Další část obsahuje popis vývoje této aplikace a řešení problémů, které mohou během něj nastat.
}\\
\par
\noindent
\textit{
Práce zároveň srovnává klasickou a Cell architekturu a popisuje nutné podmínky pro vytvoření efektivní aplikace pro \mbox{Cell/B.E.}
Dále obsahuje stručný postup instalace nejdůležitějších vývojových nástrojů.
Tento postup si klade za cíl co nejrychleji připravit vše potřebné a zkrátit tak dobu přípravné fáze tak, aby čtenář mohl začít vyvíjet pro \mbox{Cell/B.E.}
}\\

\noindent
\textbf{Klíčová slova:} \textit{programování pro Cell, multicore acceleration, IBM, PS3} \\

\pagebreak

%%%%%%%%%%%%%%%%%%%%%%%%%%%%%%%%%%%%%%%%%%%%%%%%%%%%%%%%%%%%%%%%%%%%%%%%%%%%%%%%

\noindent
\textbf{Title:} \textit{Implementation of image processing algorithms on IBM Cell} \\
\textbf{Author:} \textit{Bc. Václav Klecanda} \\
\textbf{Department:} \textit{Department of Software and Computer Science Education} \\
\textbf{Supervisor:} \textit{Mgr. Václav Krajíček} \\
\textbf{Supervisor's e-mail address:} \textit{Vaclav.Krajicek@mff.cuni.cz} \\
\textbf{Abstract:} \\

\par
\noindent
\textit{
This work summarize available information about IBM \mbox{Cell/B.E.} architecture to let the reader create a necessary overview for programming for this architecture.
Practical information are based on development of an application that implements nontrivial image processing algorithm, sparse field level set segmentation.
Next section contains description of the application development and associated problems solving.
}\\
\par
\noindent
\textit{
The work compares common and Cell B.E. architectures and describes conditions necessary for creation of an effective \mbox{Cell/B.E.} application.
The work also contains brief procedure of the most important development tools installation.
This procedure has to prepare everything necessary as fast as possible and thus to shorten the duration of the preparation phase to let the reader to start development.
}\\

\noindent
\textbf{Keywords:} \textit{CellBE programming, multi-core acceleration, IBM, PS3}

\pagebreak