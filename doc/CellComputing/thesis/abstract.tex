\raggedbottom


%%%%%%%%%%%%%%%%%%%%%%%%%%%%%%%%%%%%%%%%%%%%%%%%%%%%%%%%%%%%%%%%%%%%%%%%%%%%%%%%
\noindent
\textbf{Název práce}: \textit{Implementace algoritmů pro zpracování obrazu na IBM Cell} \\
\textbf{Autor:} \textit{Bc. Václav Klecanda} \\
\textbf{Katedra (ústav):} \textit{Kabinet software a výuky informatiky} \\
\textbf{Vedoucí diplomové práce:} \textit{Mgr. Václav Krajíček} \\
\textbf{e-mail vedoucího:} \textit{Vaclav.Krajicek@mff.cuni.cz} \\
\textbf{Abstrakt:} \\

\par
\textit{
Řešitel nastuduje dostupnou literaturu a technickou dokumentaci o programování na architektuře IBM CELL BP.
Naučí se ovládat rozhraní, prostředí a vývojové nástroje k vytváření programů pro architekturu Cell a její konkrétní implementaci (PS3, Cell BE simulátor, Cell blade server).
Zjistí, prostuduje a ověří její potenciál a možnosti pro paralelní počítání.
Osvojí si různé návrhové vzory při programování Cellu (například jak řešit nesdílenou a značně kapacitně omezenou paměť).
}

\par
\textit{
Cílem této práce je vyzkoušet některé algoritmy (registrace, segmentace) zaměřené na zpracování lékařských dat (snímky z CT, MR, rentgenu), kde se typicky pracuje s velkými objemy dat a kde je potřeba relativně rychle získat výsledky, protože diagnozy je potřeba provádět na velkých počtech pacientů.
Je vhodné provést srovnávací studii mezi implementací na klasické PC architektuře a architektuře Cell.
A to nejen co se týče rychlosti, přesnosti, ale i složitosti kódu a způsobu návrhu algoritmů jako takových.
}\\

\noindent
\textbf{Klíčová slova:} \textit{programování pro Cell, multicore acceleration, IBM, PS3} \\

\pagebreak

%%%%%%%%%%%%%%%%%%%%%%%%%%%%%%%%%%%%%%%%%%%%%%%%%%%%%%%%%%%%%%%%%%%%%%%%%%%%%%%%

\noindent
\textbf{Title:} \textit{Implementation of image processing algorithms on IBM Cell} \\
\textbf{Author:} \textit{Bc. Václav Klecanda} \\
\textbf{Department:} \textit{Department of Software and Computer Science Education} \\
\textbf{Supervisor:} \textit{Mgr. Václav Krajíček} \\
\textbf{Supervisor's e-mail address:} \textit{Vaclav.Krajicek@mff.cuni.cz} \\
\textbf{Abstract:} \\

\par
\textit{
Study available literature and technical documentation about programming on IBM Cell B.E. architecture.
Interfaces, development environments and tools for creating programs for Cell B.E. and its particular implementation Play Station 3, IBM system simulator, Cell Blade.
Find out its potential and abilities for parallel processing. Learn about design patterns.
}

\par
\textit{
Try some algorithm (registration, segmentation) aimed to medical data processing (CT, MR, X-ray images) where huge data sizes are processed and where results has to be get relatively fast because diagnosis are made for large number of patients.
It suitable to make comparison study between common PC architecture and Cell B.E. aimed to speed, precision, code complexity and ways of algorithms design itself.
} \\

\noindent
\textbf{Keywords:} \textit{CellBE programming, multi-core acceleration, IBM, PS3}

\pagebreak